\documentclass{scrartcl}

\usepackage{biblatex}
\bibliography{refs}

\usepackage{tikz-cd}

\usepackage{amssymb}
\usepackage{amsmath}
\usepackage{listings}

\lstset{ %
  basicstyle=\ttfamily,
  breaklines=true,
  captionpos=b,
  showspaces=false,
  showtabs=false,
  language=Haskell,
  escapechar=@,
  deletekeywords={Nothing,Just,False,True,putStrLn,fail,fromJust,lookup,Num,exp,free,snd,String,
  return,error,otherwise,not,show,read,Eval,Read,readsPrec,print,insert,length,id,try,Left,Right,union,
  join,Monad,Functor,Either,all,Maybe,mapM},
}

\usepackage{hyperref}
\newcommand{\lst}[1]{\lstinline!#1!}
\newcommand{\Cat}[1]{\mathbf{#1}}
\newcommand{\Op}[1]{#1^{\mathbf{op}}}
\newcommand{\Arr}[1]{#1^{\rightarrow}}
\newcommand{\iso}[0]{\cong}
\newcommand{\comma}[2]{#1 \downarrow #2}
\newcommand{\adjoint}[0]{\dashv}
\newcommand{\Hom}[3]{\mathit{Hom}_{#1}(#2,#3)}
\newcommand{\slice}[2]{#1 / #2}


\title{Category Theory Study Group}
\date{}

\begin{document}

\maketitle

\section*{Fourth Session}
For the fourth session, read 2.3. - 2.8.

\begin{enumerate}
\item
  Remember the category of directed graphs and graph homomorphism:
  The objects are graphs consisting of a set of vertices $V$ and edges $E \subseteq V \times V$.
  An arrow $f^*: (V,E) \rightarrow (V',E')$ between two graphs is a mapping between the vertices $f:V \rightarrow V'$, such that if $(v_1,v_2) \in E$ then $(f(v_1),f(v_2)) \in E'$.
  
  What is an initial/terminal object in this category.
  How many \emph{points} does the following graph have?
  \begin{center}
  \begin{tikzcd}
    G_1:
    & v_1 \ar[r] \ar[rd] & v_2 \ar[d] \\
    &                    & v_3 \ar[loop right]
  \end{tikzcd}
  \end{center}

  Prove that the previous graph is not isomorphic to the following graph by selecting a suitable ``test-object'' $X$ and show that the hom-sets have different cardinality, i.e. $|\Hom{\Graphs}{X}{G_1}| \neq |\Hom{\Graphs}{X}{G_2}|$.
  \begin{center}
  \begin{tikzcd}
    G_2:
    & r_1 \ar[r] & r_2 \ar[d] \\
    &            & r_3 \ar[loop right] \ar[ul]
  \end{tikzcd}
  \end{center}

\item What is the product graph of the following two graphs?
  \begin{center}
  \begin{tikzcd}
    G_1:                       & & G_2:       & \\
    v_1 \ar[d] \ar[loop right] & & r_1 \ar[r] & r_2 \ar[d] \\
    v_2                        & &            & r_3 \ar[loop right] \ar[ul]
  \end{tikzcd}
  \end{center}

  Consider the following graph
  \begin{center}
  \begin{tikzcd}
    G_3:
    & s_1 \ar[d] \ar[r] & s_2 \ar[d] \\
    & s_3 \ar[r]        & s_4 &
  \end{tikzcd}
  \end{center}
  and graph homomorphisms $f: G_3 \rightarrow G_1$ and $g: G_3 \rightarrow G_2$ defined by
  \[
    f(x) =
    \begin{cases}
      v_1, & x \in \{ s_1, s_2, s_3 \} \\
      v_2, & x = s_4
    \end{cases}
  \]
  and
  \[
    g(x) =
    \begin{cases}
      r_1, & x = s_1 \\
      r_2, & x = s_2 \\
      r_3, & x \in \{ s_3, s_4 \}.
    \end{cases}
  \]

  Use the UMP of the product graph to construct a unique mapping into $G_1 \times G_2$ as in the following diagram:
  \begin{center}
  \begin{tikzcd}[row sep=large]
    & G_3 \ar[dl,"f" above left] \ar[d,dashed,"{\langle f, g \rangle}"] \ar[dr,"g"] & \\
    G_1 & G_1 \times G_2 \ar[l,"\pi_1" below] \ar[r,"\pi_2" below] & G_2
  \end{tikzcd}
  \end{center}

\item
  Prove the following simple facts about products and terminal objects:
  \begin{itemize}
  \item $(A \times 1) \iso A$
  \item $(1 \times B) \iso B$
  \item $(A \times B) \times C \iso A \times (B \times C)$
  \end{itemize}
  Where have we seen these laws before?
  Do these propositions also hold for coproducts and initial objects?

  Are products commutative, i.e. $A \times B \iso B \times A$?

\item
  For any index set $I$, define the product $\prod_{i \in I} X_i$ of an $I$-indexed of objects $(X_i)_{i \in I}$ in a category, by giving a UMP generalizing that for binary products (the case $I = 2$).

  Show that in $\Cat{Sets}$, for any set $X$ the set $X^I$ of all functions $f:I \rightarrow X$ has this UMP, with respect to the ``constant family'' where $X_i = X$ for all $i \in I$, and thus
  $$
  X^I \iso \prod_{i \in I} X.
  $$

\item

  Given a category $\Cat{C}$ with object $A$ and $B$, define the category $\Cat{C}_{A,B}$ to have objects $(X,x_1,x_2)$, where $x_1:X \rightarrow A$, $x_2: X \rightarrow B$, and with arrows $f:(X,x_1,x_2) \rightarrow (Y,y_1,y_2)$ being arrows $f:X \rightarrow Y$ with $y_1 \circ f = x_1$ and $y_2 \circ f = x_2$.

  Show that $\Cat{C}_{A,B}$ has a terminal object just in case $A$ and $B$ have a product in $\Cat{C}$.
\end{enumerate}

\end{document}

%%% Local Variables:
%%% mode: latex
%%% TeX-master: t
%%% End:

%  LocalWords:  cardinality
