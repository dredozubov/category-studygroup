\documentclass{scrartcl}

\usepackage[round]{natbib}

\usepackage{tikz-cd}

\usepackage{amssymb}

\usepackage{hyperref}

\title{Category Theory Study Group}
\date{}

\begin{document}

\maketitle

\section*{Format}
Denis Redozubov asked me to help organize a study group for category theory.
As format for this study group I propose that we read about a particular concept in a category theory book or on \url{ncatlab.org}.
In addition, I will post exercises about the topics and we discuss our solutions in a group session. But ultimately I would like this group to become self-sustained:

Category theory introduces many abstractions that were derived from properties of sets and functions.
A good way to learn what these abstractions mean is to look at instances of these abstractions in different settings (i.e. in different categories).
An even better way to learn is to work out these instances yourself in a setting you are familiar with.
I encourage you to pick a category that is of interest to you (e.g. the category of database tables as objects and foreign keys as arrows) and work out what the structure of this category is as we go through the book, i.e. what are functors from and into this category, what are initial and terminal objects, etc.
If you come up with good examples, please share them with the group that everyone can learn from them.

\section*{First Session}
For the first session, read 1.1 - 1.5 of \emph{Category theory} by \cite{awodey2010category}. In addition you might want to watch recordings of the category theory lectures of Steve Awodey\footnote{\url{https://www.cs.uoregon.edu/research/summerschool/summer12/curriculum.html}} or the lectures of Bartosz Milewski \footnote{\url{https://youtu.be/I8LbkfSSR58}}.


\begin{enumerate}

\item
  Show that database schemas can be regarded as categories, where objects are tables and the arrows are foreign keys.
  What are the identities and composition operations?
  Try to find examples of categories in your own research area/work.

\item \label{ex:matrix}
  Arrows in a category are not necessarily functions.
  Define a category where the objects are positive numbers and an arrow from $n \rightarrow m$ is a $m \times n$ matrix.
  What is a suitable composition operation?

\item
  A diagram commutes if all arrows with the same domain and codomain are equal.
  What does it mean that the following diagram commutes?
  Write down all equalities between pairs of arrows and their compositions that hold.
  \begin{center}
  \begin{tikzcd}
    A \arrow[r, "f_1", bend left]\arrow[r, "f_2", bend right] \arrow[d,"f_3"] & B \arrow[d,"f_4"] \\
    C \arrow[r, "f_5", bend left]\arrow[r, "f_6", bend right] & D
  \end{tikzcd}
  \end{center}

\item
  Partial orders are categories where objects are elements and there is at most one arrow between objects $a$ and $b$ iff $a \leq b$.
  Draw a diagram of the category generated by the partial order of the power set of $\{1,2,3\}$ and subset inclusion.
  What are the identities of this categories and why does there exists a composition?
  Prove that a functor between two partial orders regarded as categories is a monotone function.

\item
  Monoids are sets/types equipped with an distinguished element called unit and an associative binary operation.
  %\footnote{Haskell includes a comprehensive library for monoids: \url{https://hackage.haskell.org/package/base/docs/Data-Monoid.html}. You might want to study the monoids $\mathtt{Any}$, $\mathtt{All}$, $\mathtt{First}$, $\mathtt{Last}$, $\mathtt{Max}$, $\mathtt{Min}$, $\mathtt{Sum}$, $\mathtt{Product}$,  $\mathtt{Endo}$, $\mathtt{Dual}$, for tuples and for functions.}
  Show that lists under the append operation form a monoid.
  Show that the $\mathtt{Option}/\mathtt{Maybe}$ type is a monoid.
  Show that the natural numbers under addition are a monoid.

  We now consider the category of monoids and their homomorphism.
  Come up with a commuting diagram between the three monoids from before.
  In other words, find at least three monoid homomorphisms between these types and show that all arrows with the same domain and codomain are equal.

\item
  The image of functors can be used to describe a specific diagram in a category.
  Define a functor with a suitable domain, whose image is the diagram of the previous exercise.
  Define a functor with a suitable domain whose image is this diagram.
  \begin{center}
  \begin{tikzcd}[row sep=large, column sep=small]
     & A \arrow[rd,"f_3"] & \\
    B \arrow[ru, "f_2"] \arrow[rr,"f_6"] & & C \arrow[d,"f_4"] \\
    D \arrow[rr,"f_5"] \arrow[u,"f_1"]  & & E
  \end{tikzcd}
  \end{center}
  How do the images of other functors with the same domain as the previous functor look like?

\item Functors transfer commuting diagrams in one category to another category.
  To this end, prove if the upper face of the following diagram commutes, so does the lower face.
  \begin{center}
  \begin{tikzcd}[row sep=small]
    &  & B \arrow[rd,"g"] & \\
    \mathbf{C} \arrow[ddd,"F"] & A \arrow[ru, "f"]\arrow[rd, "i"{below left}]  & & D \\
    &  & C \arrow[ru,"j"{below right}] & \\
    &  & FB \arrow[rd,"F g"] & \\
    \mathbf{D} & FA \arrow[ru, "F f"]\arrow[rd, "F i"{below left}] & & FD \\
    &  & FC \arrow[ru,"F j"{below right}] &
  \end{tikzcd}
  \end{center}

\item
  Show that there is a category $\mathbf{Cat}$, whose objects are categories and arrows are functors.
  What are identity functors?
  How is the composition of two functors defined?
  Prove the identity laws and that composition is associative.

\item
  An isomorphism between objects indicates, that the objects might have different structure, however, they behave the same way.
  From a perspective of a category, these objects are indistinguishable as there is the same number of arrows pointing into these objects and the same number pointing away from these objects, as we shall see later.
  For category theorists, this is a notion of equivalence of objects that is fine-grained enough and it is often not worth it to study differences between isomorphic objects.
  In this vein, prove that the relation $X \cong Y$ is an equivalence relation.

  In $\mathbf{Set}$ isomorphisms are bijective functions.
  In $\mathbf{Mon}$ isomorphisms are bijective monoid homomorphisms.
  However, in other categories, isomorphisms might be something completely different.
  What are isomorphisms in the category of matrices of exercise \ref{ex:matrix}?
\end{enumerate}

\bibliography{references}
\bibliographystyle{plainnat}

\end{document}

%%% Local Variables:
%%% mode: latex
%%% TeX-master: t
%%% End:
